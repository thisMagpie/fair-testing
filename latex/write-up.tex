\documentclass[12pt]{article}
\usepackage{natbib}
\usepackage{todonotes}
\usepackage[hmargin=1.5cm,vmargin=1.5cm]{geometry}
\usepackage{hyperref}
\usepackage{cleveref}

\crefname{table}{table}{tables}
\Crefname{table}{Table}{Tables}
\crefname{figure}{figure}{figures}
\Crefname{figure}{Figure}{Figures}
\crefname{equation}{equation}{equations}
\Crefname{Equation}{Equation}{Equation}

\title{Devising a Fair System of Trials}
\author{Magdalen Berns \\
    Supervisor: Professor G. Ackland}
\date{\today}

\setlength\parindent{0pt}
\begin{document}
\maketitle
\thispagestyle{empty}

\begin{abstract}
\noindent
\todo[inline]{describe briefly a) Why the study is important \\
                               b) the key results \\
                               c) why they are significant }
\end{abstract}

\clearpage
\tableofcontents
\thispagestyle{empty}
\clearpage
\section{Introduction}

\paragraph{Determining the most reproducible, relabable and accurate result which describes a student's
ability using the fewest number of trials possible is an area of study largly unexplored in
Higher education to date. The University of Edinburgh Teaching and assessment regulations
assertion that a 2\% margin for error exists as a possibility on a student's exam score.\cite{}}


\paragraph{Yet, there is no referenced data or citation in the regulation which indicates where such
a specific and narrow standard deviation as 2\% on test scores has been arrived at.}


\paragraph{This work sets out to determine a fair system of trials and ultimately to determine
whether or not the University of Edinburgh is justified in making the assumption that
a 2\% error on exam grades exists by comparing concrete testing methods against physical
examination data provided by the school of Physics and Astronomy\todo[inline] {finish this paragraph}}

\section{Theory and Algorithms}

\paragraph{Exam result data cannot be catagorised because it is theoretically possible to achieve
any grade between 0-100. That means a parametic test is preferable for significance
 testing of the data so that the results set can be transformed into a normal distribution.}

\paragraph{The sum of square deviations of data points $x_i$ from the expectation model mean value $\mu$
 determined by \cref{eq:mu} is given by \cref{eq:ssq}}

\begin{equation}
\mu = \frac{x_1 + x_2 + \cdots + x_n}{n} = \frac{1}{N}\sum_{i}^{N} x_i
\label{eq:mu}
\end{equation}

\begin{equation}
\Sigma{(x_i-\mu)^2}
\label{eq:ssq}
\end{equation}

The standard deviation of a population variance is given by \cref{eq:std-dev} 

\begin{equation}
\sigma=\sqrt{\frac{\Sigma{(x_i-\mu)^2}}{N}}
\label{eq:std-dev}
\end{equation}

\subsection{Gaussian Distribution}

\subsection{Point Spread Function}

\subsection{Central Limit Theorem}

\section{Method and Code Design}

The code design was written in java 7 and the graphs were written using pgfplot
package in latex \todo[inline]{ use cool latex font}

\subsection{Examination Difficulty Level}

\paragraph{The first system assumes that the student is flawless and that any standard deviation in grades is due to a flaw in in the exam or teaching with a standard deviation of $2\%$ which inadvetantly leads to an exam which is either too easy or too hard for the student's level of physics skill which leads to a deviation from the mean of $\pm{2\%}$. \cite{}}

\subsection{Student Ability To Do Well in Exam}

The second system assumes that the lecturer and his or her exam is flawless in its execution and that any standard deviation in grades is due to a flaw or exceptional ability in the student $22.6\%$.\cite{}

%\subsection{Testing the simple \enquote{One Parameter} Model}

\section{Results and Discussion}

\section{Conclusion}
\bibliography{fairtest}
\bibliographystyle{unsrt}
\end{document}
